\documentclass{article} % For LaTeX2e
\usepackage{nips15submit_e,times}
\usepackage{hyperref}
\usepackage{url}
\usepackage{listings}
\usepackage{float}
\usepackage{amsmath,amsthm,amssymb,graphicx,mathtools,tikz}

\lstset{ %
	language = C, 
	language = C++,                		
	language=Matlab,
	language=Python,
	basicstyle=\ttfamily\footnotesize,        
	keywordstyle=\color{blue}\ttfamily,
	stringstyle=\color{red}\ttfamily,
	commentstyle=\color{magenta}\ttfamily,
	numberstyle = \footnotesize,      	
	numbers = left,                   		
	stepnumber = 1,                   	
	numbersep = 5pt,                  	
	backgroundcolor = \color{white},  	
	showspaces = false,               		
	showstringspaces = false,      		
	showtabs = false,                		
	frame = single,          			
	tabsize = 4,         				
	captionpos=b,          				
	breaklines=true,        			
	breakatwhitespace=false,    		
	escapeinside={\%*}{*)}     
}

\title{CSE 202: Algorithm Design and Analysis \\ Homework Assignment 1}

\author{
  Mingyang Wang, Ding Wang, Zhimin Zhou\\
  Department of Computer Science\\
  University of California, San Diego\\
  \texttt{\{miw092, diw005, zhz249\}@eng.ucsd.edu}\\
  A53100579 A53089251 A53089795\\
}

\newcommand{\fix}{\marginpar{FIX}}
\newcommand{\new}{\marginpar{NEW}}

\nipsfinalcopy % Uncomment for camera-ready version

\begin{document}

\maketitle
\section*{Problem 1}

\begin{enumerate}
\item

\item 

  Proof by contradiction. Assume there is a MST $T$ and $e$ is not in it. Then
  we can add $e$ into $T$. This will lead to a cycle $C$ which contains $e$.

  We could always remove any other edge from cycle $C$ since every other edge in
  $c$ has a bigger weight than $e$. After that, we will still have a tree but this
  tree will have a smaller weight compared with $T$. Therefore $T$ is not a MST
  and we have found a contradiction.

\item

  False. We could have this cycle $C$ with 4 nodes. Each edge has weight bigger
  than 100. And there is another path including one more node between each 2 nodes
  which only cause 2 more weight. In this situation, the MST will have no edge
  from $C$.

\item

  Same as problem 3.

  False. We could have this cycle $C$ with 4 nodes. Each edge has weight bigger
  than 100. And there is another path including one more node between each 2 nodes
  which only cause 2 more weight. In this situation, the MST will have no edge
  from $C$.

\item

  Proof by contradiction. Assume $T_1$ and $T_2$ are two different MSTs. Lets
  say $e_1$ is the edge with smallest weight that is in $T_1$ but not in $T_2$.
  Then we add $e_1$ into $T_2$ and we get a cycle. Therefore one edge in the
  cycle we call it $e_2$ is not in $T_1$.

  We know that the weight of $e_2$ is bigger than the weight of $e_1$, and
  therefore we know $T_2$ = $T_1$ $\cup$ $\{e_2\}$ $\setminus$ $\{e_1\}$ has a
  total weight bigger than $T_1$. Therefore it is not a valid MST. Contradiction
  found.

\end{enumerate}

\newpage
\section*{Problem 2}
\begin{enumerate}
\item 1, 3, 3, 1
\item 
  \begin{enumerate}
  \item 
    Assume that $i \leq d_1 + 1, j > d_1 + 1$ and there is a link between $v_1$
    and $v_j$. Then we can have $(v_1, v_i) \notin E and (v_1, v_j) \in E$.

    Another thing is that we know that the degree of $v_i$ should be bigger than
    or equal to $v_j$, and since we already have $(v_1, v_i) \notin E and (v_1,
    v_j) \in E$. We know that there must be another node $u$ that $(u, v_i) \in
    E and (u, v_j) \notin E$.

  \item 
    The changes need to be made could be from part(a). Namely we coud remove the
    two edges $(u, v_i), (v_1, v_j)$ and replace with another two edges $(v_1,
    v_i), (u, v_j)$.
    
    By doing this, we could have each node degree unchanged but generated the
    required edge $(v_1, v_i)$.

  \item 
    By combining prat(a) and part(b). We could know that for each neighbor of
    $v_1$ that is not belongs to $v_2 \text{to} v_{d_1+1}$, we would have
    another $u \in V$ that $(u, v_i) \in E and (u, v_j) \notin E$, $(v_1, v_i)
    \notin E and (v_1, v_j) \in E$. Therefore according to part(b), we could
    always transform the above pair and get rid of edge $(u, v_j)$ and get a new
    edge $(u, v_i)$.

  \end{enumerate}

\item 
  The algorithm is described below:


\end{enumerate}
\newpage
\section*{Problem 3}

\newpage



\end{document}